% Copyright (C)  2013  Chen Ruichao <linuxer0x@163.com>.
% Permission is granted to copy, distribute and/or modify this document
% under the terms of the GNU Free Documentation License, Version 1.3
% or any later version published by the Free Software Foundation;
% with no Invariant Sections, no Front-Cover Texts, and no Back-Cover Texts.
% A copy of the license is included in the section entitled "GNU
% Free Documentation License".

% XXX \setcounter{secnumdepth}{0} is a better idea than \section*, but
%     \section doesn't work well with some packages we're using.

\documentclass{article}
\usepackage{CJK}
\usepackage{setspace}
\usepackage{indentfirst}
\usepackage[colorlinks=true]{hyperref}
\usepackage{verbatim} % seems we don't have to \usepackage it to use the verbatim environment
\usepackage{graphicx}
\usepackage[rightcaption]{sidecap}
\usepackage{caption}
\usepackage{multicol}
\usepackage{xmpincl}
\includexmp{rc/license}

\begin{document}
\begin{CJK*}{UTF8}{gbsn}

    \pagenumbering{gobble}
    \setcounter{secnumdepth}{0}
    \setlength{\parindent}{2em}

    \title{2013-12-21周末赛}
    \author{陈睿超 \texttt{<linuxer0x@163.com>}}
    \date{}
    \maketitle
    \newpage






    \section*{\centering对齐(alignment)}
    \subsection*{题目描述}
    给出$N$种物件,每种物件个数无限. 对于第$i$种物件,每个物件拥有一个高度$A_i$. 将物件堆叠起来,求有多少种方法能堆到恰好总高度为$H$.

    同种物件完全相同,顺序任意调换后仍然视为同种方案. 不同种物件间顺序调换视为不同方案. 镜像的方案视为不同方案(因为是自底向上堆叠).


    \subsection*{输入格式}
    第一行一个整数$N$.

    第二行有$N$个整数,$A_1, A_2, A_3, \dots, A_N$.

    第三行一个整数$H$.

    \subsection*{输出格式}
    仅一行,一个整数,表示方案数.

    答案$\bmod \; 100000007$后输出.

    \subsection*{样例数据}
    \begin{spacing}{0.8}
    \begin{multicols}{3}
    \raggedcolumns

    \subsubsection*{输入数据}
    \begin{verbatim}
    2
    1 2
    3
    \end{verbatim}

    \subsubsection*{输出数据}
    \begin{verbatim}
    3
    \end{verbatim}

    \subsubsection*{样例解释}
    {\footnotesize{
    共有$3$种方案:

    aaa

    ab

    ba
    }}

    \columnbreak

    \subsubsection*{输入数据}
    \begin{verbatim}
    2
    1 1
    3
    \end{verbatim}

    \subsubsection*{输出数据}
    \begin{verbatim}
    8
    \end{verbatim}

    \subsubsection*{样例解释}
    {\footnotesize{
    共有$8$种方案:

    aaa

    aab

    aba

    abb

    baa

    bab

    bba

    bbb
    }}

    \columnbreak

    \subsubsection*{输入数据}
    \begin{verbatim}
    5
    1 2 3 2 1
    931
    \end{verbatim}

    \subsubsection*{输出数据}
    \begin{verbatim}
    38789637
    \end{verbatim}
    \end{multicols}
    \end{spacing}

    \subsection*{数据范围}
    $1 \leq N \leq 1000$

    $1 \leq A_i \leq 10$

    $0 \leq H \leq 10^{12}$

    \newpage






    \section*{\centering项目(project)}
    \subsection*{题目描述}
    绵羊正在用两种语言编写一个项目.

    这个项目共有$N$个模块,每个模块必须只用一种语言来编写. 用语言A编写的费劲程度为$A_i$,用语言B编写的费劲程度为$B_i$.

    有$M$对模块需要进行交互. 若两个模块用同一种语言编写,则无须编写中间代码. 若两个模块使用不同的语言,则需要编写胶水代码来允许两个模块互相调用. 若一个模块用语言A编写,另一个用语言B编写,则额外的费劲程度为$C_j$;若反之,则额外的费劲程度为$D_j$%
    \footnote{两个模块之间可能有多处需要进行交互,此时费劲程度叠加}%
    .

    显然绵羊是一个很懒的人,所以他希望总的费劲程度尽可能地小. 请你求出最小的总费劲程度.

    \subsection*{输入格式}
    第一行两个数$N$, $M$

    接下来$N$行,每行两个整数$A_i$, $B_i$,表示第$i$个模块用语言A和语言B编写的费劲程度.

    接下来$M$行,每行四个整数$x$, $y$, $C_j$, $D_j$. $C_j$表示第$x$个模块用语言A编写,第$y$个模块用语言B编写的费劲程度;$D_j$表示第$x$个模块用语言B编写,第$y$个模块用语言A编写的费劲程度.

    \subsection*{输出格式}
    仅一行,一个整数,表示最小总费劲程度.

    \subsection*{样例数据}
    \begin{multicols}{2}
    \subsubsection*{输入数据}
    \begin{verbatim}
    3 1
    1 10
    2 10
    10 3
    2 3 1000 1000
    \end{verbatim}

    \columnbreak

    \subsubsection*{输出数据}
    \begin{verbatim}
    13
    \end{verbatim}
    \end{multicols}

    \subsection*{数据范围}
    $1 \leq N \leq 20000$

    $1 \leq M \leq 200000$

    $1 \leq A_i, B_i, C_j, D_j \leq 10000$

    \newpage





    \section*{\centering怪树树(toy)}
    \subsection*{题目描述}
    绵羊有一棵怪树树. 每过一段时间,这棵树的枝条就会自行脱落. 而绵羊每周末的重要活动之一就是把这棵树拼起来.

    周末快要到了,怪树树又变成了一片森林. 但是绵羊忙于补番,因此这周末的工作就拜托给你了. 为了确信你有认真干活,绵羊会时不时地询问某两个结点的LCA.

    嗯,就是这样.

    \subsection*{输入格式}
    第一行一个整数$N$,表示一开始有$N$个互不相连的结点,最终它们会被接成一棵有根树. 结点被编号为$1 \sim N$.

    接下来有若干行,每行有三个整数,表示一个操作.

    \begin{itemize}
        \item M $u$ $v$   $\quad$   绵羊要求你将$v$所在的树的根连到$u$上,成为$u$的子结点. 若两个结点已经位于同一棵树中,则无视绵羊的要求.
        \item Q $u$ $v$   $\quad$   绵羊想知道结点$u$和结点$v$现在的LCA.
    \end{itemize}

    $N$个结点被连成一棵有根树后,操作结束.

    \subsection*{输出格式}
    对于每个询问,输出一行,仅一个整数,表示所查询的两个结点的LCA. 若LCA不存在,输出$-1$.

    \subsection*{样例数据}
    \begin{multicols}{2}
    \subsubsection*{输入数据}
    \begin{verbatim}
    5
    Q 3 4
    Q 3 3
    M 1 2
    Q 1 2
    Q 2 1
    M 2 3
    M 1 4
    Q 3 4
    M 1 5
    \end{verbatim}

    \columnbreak

    \subsubsection*{输出数据}
    \begin{verbatim}
    -1
    3
    1
    1
    1
    \end{verbatim}
    \end{multicols}

    \subsection*{数据范围}
    对于\phantom{0}40\%的数据,$1 \leq N \leq 2000$,连接操作不超过$10000$次

    对于100\%的数据,$1 \leq N \leq 100000$,连接操作不超过$1000000$次,询问操作不超过 $1000000$ 次

    \newpage







    \newpage
    \LaTeX 苦手,么介意
    \begin{SCfigure}[2.0]
        \centering
        \input{rc/by-sa.pdf_tex}
        \caption*{`2013-12-21周末赛' 由 陈睿超 创作,采用 \href{http://creativecommons.org/licenses/by-sa/4.0/}{知识共享 署名-相同方式共享 4.0 国际 许可协议}进行许可。}
    \end{SCfigure}

\end{CJK*}
\end{document}
